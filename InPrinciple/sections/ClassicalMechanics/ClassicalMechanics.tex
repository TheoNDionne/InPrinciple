\chapter{Classical Mechanics}

\begin{center}
    \begin{minipage}{0.5\textwidth}
    	\textit{\red{
            Some sort of quote could go here easily and be beautifully typeset. For example, consider: Lorem ipsum dolor sit amet.
     }}
    \end{minipage}
\end{center}


\section{
    Physical motivation
}

Simply stated, classical mechanics is the physical theory of the world as we experience it. The main goal of this chapter is to state some reasonable \textit{axioms} (statements that form the hypotheses of a mathematical theory) inspired by our experience of Nature (which is inherently confined to our scale) which are then used to derive the theory known as classical mechanics. Although this procedure may be regarded as superfluous or \enquote{overkill} for this well known science, it is important to realize \textit{how} physical laws emerge from assumptions, since this is how the hypotheses underpinning the ideas are tested. In principle then, if the theory is derived with rigour from the axioms, this procedure is sufficient to carry out proper science. Hence, this chapter will also serve as inspiration and even a template for more complex fields for which the physical conclusions are far less obvious from the start.

The main approach in establishing the axioms will be to look for regularity; once it has been identified what \textit{type} of phenomena is being studied, one can then search for patterns which characterize all the phenomena of a given class. In this first case, we will focus on how objects move on our scale.

To start, we notice that objects can be in different \textit{places}. This we will take to be a basic property of what we call \textit{space}. Two separate places are related by the intuitive notion of a \textit{distance}. From this heuristic notion of distance, we can add further nuance: not all objects that are the same distance away from a reference object are in the same place, there is thus also a notion of what we call \textit{direction}. Taken together, we thus see that two objects are related by a notion of relative position.

\red{
Although we could refer exclusively to the distance between objects, it would become extremely difficult to keep track of a collection of anything over a few. Hence, a useful trick is to set an \textit{origin}: a point of no physical significance, but a reference point in space whose distance from all other points is the way in which \textit{position} is defined.

\begin{definitionbox}{Position}{def:position}
    \red{The position of any point is its distance from the origin. The vector that encodes position is labeled $\ve{q}\in\mathbb{R}^3$. The components of $\ve{q}$ are called the coordinates.}
\end{definitionbox}
}

\begin{axiombox}{Relative position}{ax:rel_position}
\red{relative position is not the right vocab... workshop this, maybe the correct axiom is then space???}

Let $P_1, P_2\in\mathbb{R}^3$ be two points in space with $\ve{r}_1,\ve{r}_2\in\mathbb{R}^3$ their respective coordinates. Their relative position (which encodes both the notion of distance and direction) is defined as the vector $\ve{r}_{2\leftarrow1} \equiv \ve{r}_2 - \ve{r}_1$. Note that this property is entirely independent of the choice of origin and is thus intrinsic (contrary to position itself).
\end{axiombox}

Of course, we also know that things \textit{happen}. For example, you were born, now you live and one day you will eventually die. We could then say that there is a natural order to events.

\begin{axiombox}{Chronology}{ax:chronology}
	There exists an order in which things happen.
\end{axiombox}

\red{come up with a smooth transition into duration}

\begin{definitionbox}{Time}{def:time}
    \red{The definition of time.} Once a temporal origin is chosen, the number that encodes the time is called $t\in\mathbb{R}$.
\end{definitionbox}

\red{basically, I am realizing that Space->Position->Relative position is the same as Chronology->Time->Duration... make this parallel extremely clear and crisp. I want to enstore early on that space and time are very close cousins with the exception of the arrow of time.}

\red{This is where there is a figure of some trajectory... this also serves as inspo for what comes next} From \red{figure}, one can see that we can keep track of how the position of an object changed with respect to time. This can then \red{inspire determinism, the fact that there is a predetermined path for matter.}. An important ingredient is the notion that \enquote{things happen for a reason}. If the classical path is predetermined, then if we have enough information about the path at one given time $t$, then we should be able to predict the future behaviour. This can be related to causality.

\red{Notion of a classical trajectory}

\begin{axiombox}{Determinism}{ax:determinism}

\end{axiombox}

\begin{axiombox}{Causality}{ax:causality}

\end{axiombox}

\red{
    Speech that brings up the idea behind locality
}

\begin{axiombox}{Locality}{ax:locality}

\end{axiombox}

A consequence of locality in time tells us that no time derivative above the second degree shall be a necessary ingredient in classical theory. Hence, since the path is supposed to be preditermined and that it is causal, hence, deducable from what has preceded, we can assume that the path can be determined from time local ingredients. This is to say that there is some differential equation of \textit{at most} second order. Hence, we arrive at a first precise statement.

\begin{corollarybox}{Equation of Motion}{cor:EoM}
    The classical trajectory of an object in classical mechanics must, from the axioms above, obey a second order differential equation.
    \al{
        \mathcal{N}[\ve{q}(t),\dot{\ve{q}}(t),\ddot{\ve{q}}(t),t] = \mathcal{N}(t) = 0 \label{eq:EoM}
    }
\end{corollarybox}

The above corollary \red{box} then tells us that for one given time $t=t_i$ knowledge of the position $\ve{q}(t_i)$ and velocity $\dot{\ve{q}}(t_i)$ fixes the classical trajectory\footnote{
    This is a result from the theory of differential equations that a differential equation of second order requires two boundary conditions to fix the solution entirely [\red{name of the theorem and reference}]
}. In principle, this is all we need theoretically to \textit{solve} classical mechanics on the mathematical front.

\begin{axiombox}{Galilean invariance}{ax:GalInvar}
    \red{Frames moving at constant velocity are physically equivalent.}
\end{axiombox}

\red{Some spiel about fixing instead position at two different paths.}
Notice aswell that the existence of a differential equation \eqref{eq:EoM} implies necessarily the existence of its vanishing integral
\al{
    \mathcal{I}[\ve{q}, \dot{\ve{q}};t_i,t_f] = \Int[t_i][t_f] \mathcal{N}[\ve{q},\dot{\ve{q}},\ddot{\ve{q}},t] = 0
}

\section{The functional approach}

\subsection{The action and the lagrangian}

In the interest of generality, we may adopt the functional approach from the very beginning given its generality.

This functional $\mathcal{I}$ is slightly undesirable as it depends formally on the velocity aswell, which is redundant given the fact that we already possess two positions. Hence, we may integrate with respect to the position to obtain:
\al{
    \mathcal{S}[\ve{q}] = \int\mathcal{D}[\ve{q}]\mathcal{I}[\ve{q}, \dot{\ve{q}}]
}
This integral is a constant with respect to the path when evaluated for the classical trajectory $\mathcal{S}[\ve{q}_c] = C$ and is called the \textit{action}. This implies that the classical trajectory obeys the following condition:
\al{\cboxed{
    \delta\mathcal{S} = 0
}}
where $\delta$ signifies the total variation. We may now wonder what this condition implies on the integrand $\mathcal{L}$,\red{whoah, I haven't even introduced the procedure completely...its about guesswork and foregiveness...} this procedure is a standard step in the calculus of variations:
\al{
    \delta\mathcal{S} &= \Int[t_i][t_f]\D{t}\,\delta\mathcal{L}\notag\\
    &= \Int[t_i][t_f]\D{t}\,\cro{\ddf[\mathcal{L}]{\ve{q}}\cdot\delta\ve{q} + \ddf[\mathcal{L}]{\dot\ve{q}}\cdot\delta\dot\ve{q}}\notag\\
    &= \Int[t_i][t_f]\D{t}\,\cro{\ddf[\mathcal{L}]{\ve{q}} - \dd{t}\ddf[\mathcal{L}]{\mkern1.5mu\dot\ve{q}}}\cdot\delta\ve{q} + \cro{\ddf[\mathcal{L}]{\mkern1.5mu\dot\ve{q}}\cdot\delta\ve{q}}_{t=t_i}^{t=t_f}\notag\\
}

Two remarks are in order. First, we have used the fact that trajectory and velocity are explicitely linked such that $\delta\dot{\ve{q}} = \dd{t}\delta\ve{q}$. Second, the variation of the path vanishes by definition at the boundaries of the integral, hence, the second term above is always zero.

Since the variation is assumed to be arbitrary, we can see that the action only vanishes for any choice of boundary if:
\al
This condition is known as the Euler-Lagrange equation and is a fundamental result of variational calculus.

\subsection{Redundance in the lagrangian formalism}

Notice that the condition that the variation of the action vanishes for the physical trajectory allows for some redundance. For example,
\al{
    \mathcal{S}'[\ve{q}] &= \alpha\mathcal{S}[\ve{q}] + C\notag\\
    \delta\mathcal{S}'[\ve{q}] &= \alpha\delta\mathcal{S}[\ve{q}] = 0
}
It is then seen that $\mathcal{S}$ and $\mathcal{S}'$ yield exactly the same path. The case for the lagrangian is even less constraining:
\al{
    \mathcal{L}'[\ve{q}, \dot{\ve{q}};t] &= \alpha\mathcal{L}[\ve{q}, \dot{\ve{q}};t] + \dd[\mkern-1.5mu f]{t}\notag\\
    \Int[t_i][t_f]\D{t}\,\mathcal{L}'[\ve{q}, \dot{\ve{q}};t] &= \alpha\Int[t_i][t_f]\D{t}\,\mathcal{L}[\ve{q}, \dot{\ve{q}};t] + f(t_f) - f(t_i)\notag\\
    \mathcal{S}'[\ve{q}] &= \alpha\mathcal{S}[\ve{q}] + f(t_f) - f(t_i)\notag\\
}
These lagrangians yield exactly the same physical solution as they are stationnary for the same trajectory. It should also be checked that this same redundance is reflected in the Euler-Lagrange condition.

\section{Equations of motion of a particle}

Part of the magic of the Lagrangian formalism is that it allows for one to obtain the equations of motion based on a guess for the lagrangian. This \enquote{educated guess} is often based on the grounds of \textit{symmetries} and \textit{invariance}. This procedure can be generalized in many ways and forms the backbone of a great part of modern physics.

\subsection{Free particle}

A free pointlike particle has to be one of the simplest cases to analyze. By free here, I mean that the particle is in a perfect unchanging vacuum. We may now pare down the form of the lagrangian until we find the equations of motion of a free particle.

Do we expect the behaviour of the particle to change based on \textit{where} it is? How about \textit{when}? More subtly, how about the direction it moves in? The answer to all of these questions is no. By supposition, all points in space and time are identical. The last thing on which the lagrangian may depend upon is the magnitude of the velocity $\abs{\dot{\ve{q}}}$. Given the definition of the norm of a vector in euclidean space $|v|\equiv\sqrt{\ve{v}\cdot\ve{v}}$, we can say then that the lagrangian depends only on the inner product $\ve{v}\cdot\ve{v}= \ve{v}^2$
\al{
    \mathcal{L} = \mathcal{L}[\ve{v}^2]
}

\subsection{Constrained particle}

\red{Alla Landau}

\section{Continuous symmetries and conserved quantities}

One of the most elegant results in the lagrangian formalism is Noether's first theorem. It states that a continuous symmetry implies a conserved quantity and vice-versa.
