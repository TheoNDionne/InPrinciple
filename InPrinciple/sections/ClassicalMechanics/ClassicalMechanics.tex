\chapter{Classical Mechanics}

\section{
    Setup of classical mechanics
}

Classical mechanics is the physics of the world as we experience it. \red{Go on to explain how the world around us has a certain a pattern of regularity, we will use these observations to formulate principles that later become axioms of a theory that we develop AKA, these axioms are the hypotheses of our scientific theory}.

We notice that objects for example can be in different \textit{places}. Two separate places are related by the intuitive notion of a \textit{distance}.

\begin{axiombox}{Distance}{ax:distance}
    The \textbf{distance} is the measure of \red{[...]}. \red{Should this be direction?}
\end{axiombox}

Although we could refer exclusively to the distance between objects, it would become extremely difficult to keep track of a collection of anything over a few. Hence, a useful trick is to set an \textit{origin}: a point of no physical significance, but a reference point in space whose distance from all other points is the way in which \textit{position} is defined.

\begin{definitionbox}{Position}{def:position}
    The position of any point is its distance from the origin. The vector that encodes position is labeled $\ve{q}\in\mathbb{R}^3$. The components of $\ve{q}$ are called the coordinates.
\end{definitionbox}

Of course, we also know that things \textit{happen}. Indeed, we have an intuitive notion of \textit{after}.

\begin{axiombox}{Chronology}{ax:chronology}
	There exists an order in which things happen. \red{should this be duration? Needs to be on equal footing as the space}
\end{axiombox}

\begin{definitionbox}{Time}{def:time}
    \red{The definition of time.} Once a temporal origin is chosen, the number that encodes the time is called $t\in\mathbb{R}$.
\end{definitionbox}

\red{This is where there is a figure of some trajectory... this also serves as inspo for what comes next}

\begin{axiombox}{Determinism}{ax:determinism}

\end{axiombox}

\begin{axiombox}{Causality}{ax:causality}

\end{axiombox}

\begin{axiombox}{Locality}{ax:locality}

\end{axiombox}

\red{Notion of a classical trajectory}

\begin{corollarybox}{Equation of Motion}{cor:EoM}
\red{change color to match the axioms.}
\end{corollarybox}

\begin{axiombox}{Galilean invariance}{ax:GalInvar}

\end{axiombox}

\section{The functional approach}
\red{}

\section{The equations of motion}

\subsection{Free particle}

\subsection{Interacting particle}
