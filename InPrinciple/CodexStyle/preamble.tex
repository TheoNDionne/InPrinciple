%%%%%%%%%%%%%%%%%%%%%%%%%%%%%%%%%%%%%%%%%%%%%%%%
%----------------Language Support---------------
\usepackage[french,english]{babel} % language support priority language last

%%% French renamings
\addto\captionsfrench{\renewcommand\frenchtablename{{\FBfigtabshape Tableau}}} % rename table name (french)
\addto\captionsfrench{%
  \renewcommand{\refname}{Bibliographie}%
}

%--------------------Colors---------------------
\usepackage[x11names]{xcolor} % adds in typical color features

%%% Choose color theme
\usepackage{CodexStyle/colorthemes/colors_marine} % choose a style

%---------------General packages----------------
\usepackage{setspace}        % line spacing
\usepackage{cite}            % in text citations
\usepackage{microtype}       % small typography tweaks
\usepackage{graphicx}		 % figure insertion
\usepackage{subfig}			 % allows for subfigures MIGRATE TO A MODERN PACKAGE MAYBE
\usepackage{geometry}		 % document geometry
\usepackage{fancybox}		 % boxes, frames etc
\usepackage{url}             % typographically sound URLs
\usepackage{float}           % Fixes hacky geometry problems
\usepackage[
format    = hang,
margin    = 5mm,
font      = small,
labelfont = bf,
labelsep  = space
]{caption}                   % tweak caption layout and format
\usepackage{csquotes}        % For quotes
% \newcommand\itsym{$\bullet$} % symbole pour les listes

\usepackage{hyperref}[
    backref     = page,
    pagebackref = true,
    hyperindex  = true,
    bookmarks   = true,
    pdfa
]

% !!!!!!!!!!!!!!!!!!!!!!!!!!!!!!!!!!!!!!!!!!!!!!!!!
% \usepackage[a-2b,mathxmp]{pdfx}[2018/12/22] % ???
% !!!!!!!!!!!!!!!!!!!!!!!!!!!!!!!!!!!!!!!!!!!!!!!!!

% options PDF
\hypersetup{
    colorlinks=true,         % colorise les liens
    breaklinks=true,         % permet le retour la ligne dans les liens trop longs
    urlcolor=URLColor,       % couleur des hyperliens (doit inclure x11names dans xcolor ci-dessus)
    linkcolor=LinkColor,     % couleur des liens internes
    citecolor=CiteColor,     % couleur des liens de citation
    bookmarksopen=true,      % ouvre les signets PDF au départ
}

%---------------CodexStyle layout---------------
% Switches between longform and shortform automatically
\makeatletter
\@ifclassloaded{book}{
    \usepackage{CodexStyle/theme/longform_layout} % load longform
}{
\@ifclassloaded{article}{
    \usepackage{CodexStyle/theme/shortform_layout} % load shortform
}{
    \PackageError{mydoc}
        {Unsupported document class!}
        {`book` -> longform, `article` -> shortform}
}}
\makeatother

%------------------Math Setup-------------------
\usepackage{amsmath}          % math environments
\usepackage{mathtools}        % tools for math formating
\usepackage[nice]{nicefrac}   % nicer fractions
\usepackage{cancel}           % allows to scratch expressions.
\usepackage{slashed}          % allows to slash individual characters.
\usepackage{xargs}            % better handling of optional arguments for commands
\usepackage{braket}           % convenient Dirac notation
\usepackage{empheq}           % colored boxes in math env
\usepackage[most]{tcolorbox}  % adds colorful box environments

%%% Import custom math macros
\usepackage{CodexStyle/math_macros} % custom latex macros (macros.sty)

%------------Fonts (Body and Math)--------------
\usepackage{fontspec} % Finer font selection (requires Lua/XeLaTeX)
\usepackage{unicode-math} % Finer math font selection

%%% Main font switches based on documentclass
\makeatletter
\@ifclassloaded{book}{
    \setmainfont{EB Garamond}[ % EB Garamond
        Ligatures   = TeX, % ligatures
        OpticalSize = On, % optical weight adjustment
        % Numbers     = OldStyle, % can be a little hard to read
        SmallCapsFeatures = {LetterSpace=5} % Adds spaces for small caps
    ]
}{
\@ifclassloaded{article}{
    \setmainfont{Libertinus Serif}[ % Libertinus in short documents
        Ligatures = TeX,
        Kerning   = On,
    ]
}{
    \PackageError{mydoc}
        {Unsupported document class!}
        {`book` -> longform, `article` -> shortform}
}}
\makeatother

%%% Math fonts (Libertinus and Latin Modern)
\setmathfont{Libertinus Math}[ % nice neutral (near) universal math font
    Scale=MatchLowercase % set scale
]
\setmathfont{Latin Modern Math}[ % A few swaps and additions
    range = {\dagger, \ddagger, \mathcal}, % less dramatic daggers and distinct mathcal
    Scale=MatchLowercase % set scale
]

%--------------------Tables---------------------
\usepackage{array} % tabular functions
\usepackage{dcolumn} % allows for aligning of values wrt to the decimal
\newcolumntype{C}{>{$\displaystyle}c<{$}} % centered math column
\newcolumntype{L}{>{$\displaystyle}l<{$}} % left aligned math column
\newcolumntype{R}{>{$\displaystyle}r<{$}} % right aligned math column
\renewcommand{\arraystretch}{1.5} % vertical spacing of tables

%%% UNDER CONSTRUCTION %%%
%-------------------------------------------------------------------------------
% Figures TiKz !!!TODO FIX THIS!!!
\usepackage{tikz}
\usetikzlibrary{
	calc,
	patterns,
	positioning,
    external,
    shapes,
    fit,
    backgrounds,
    arrows.meta,
	positioning,
    decorations,
    decorations.pathmorphing,
    decorations.markings,
    shapes.geometric,
    arrows
}
% \tikzexternalize[prefix=figures/pdf/]

\usepackage{pgfplots}
\pgfplotsset{compat=1.16}
\pgfdeclarelayer{background}
\pgfdeclarelayer{foreground}
\pgfsetlayers{background,main,foreground}

\usepackage{CodexStyle/theme/tikzstyles}
%%%%%%%%%%%%%%%%%%%%%%%%%%%%%%%%%%%%%%%%%%%%%%%%
